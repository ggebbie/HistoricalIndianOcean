% Use only LaTeX2e, calling the article.cls class and 12-point type.

\documentclass[12pt]{article}

% Users of the {thebibliography} environment or BibTeX should use the
% scicite.sty package, downloadable from *Science* at
% www.sciencemag.org/about/authors/prep/TeX_help/ .
% This package should properly format in-text
% reference calls and reference-list numbers.

\usepackage{scicite}

% Use times if you have the font installed; otherwise, comment out the
% following line.

\usepackage{times}

% The preamble here sets up a lot of new/revised commands and
% environments.  It's annoying, but please do *not* try to strip these
% out into a separate .sty file (which could lead to the loss of some
% information when we convert the file to other formats).  Instead, keep
% them in the preamble of your main LaTeX source file.

\usepackage{graphicx}
\usepackage{verbatim}
\usepackage{amsmath}
%\usepackage[pdfborder={0 0 0},colorlinks=false,linkbordercolor={0 0 0},urlbordercolor={0 0 0}]{hyperref}
%\usepackage[dvips, colorlinks=false, pdfborder={0 0 0}, urlcolor=blue, linkbordercolor={0 0 0},urlbordercolor={0 0 0}]{hyperref}
%\usepackage{url}
\DeclareMathOperator{\sinc}{sinc}

% The following parameters seem to provide a reasonable page setup.

\topmargin 0.0cm
\oddsidemargin 0.2cm
\textwidth 16cm 
\textheight 21cm
\footskip 1.0cm


%The next command sets up an environment for the abstract to your paper.

\newenvironment{sciabstract}{%
\begin{quote} \bf}
{\end{quote}}


% If your reference list includes text notes as well as references,
% include the following line; otherwise, comment it out.

\renewcommand\refname{References and Notes}

% The following lines set up an environment for the last note in the
% reference list, which commonly includes acknowledgments of funding,
% help, etc.  It's intended for users of BibTeX or the {thebibliography}
% environment.  Users who are hand-coding their references at the end
% using a list environment such as {enumerate} can simply add another
% item at the end, and it will be numbered automatically.

\newcounter{lastnote}
\newenvironment{scilastnote}{%
\setcounter{lastnote}{\value{enumiv}}%
\addtocounter{lastnote}{+1}%
\begin{list}%
{\arabic{lastnote}.}
{\setlength{\leftmargin}{.22in}}
{\setlength{\labelsep}{.5em}}}
{\end{list}}


% Include your paper's title here

\title{Historical Indian Ocean Temperature Change}

\author
{Geoffrey Gebbie$^{1\ast}$ \\
\\
\normalsize{$^{1}$Department of Physical Oceanography, Woods Hole Oceanographic Institution,}\\
\normalsize{360 Woods Hole Rd., Woods Hole, MA 02543, USA}\\ 
\normalsize{$^\ast$To whom correspondence should be addressed; E-mail:  ggebbie@whoi.edu.}
}

% Include the date command, but leave its argument blank.

\date{}
% start citations at the end of main body numbers. also, use original reference number from main body if a citation is duplicated (haven't figured out how to do this)
\usepackage{xpatch}
\newcounter{mybibstartvalue}
\setcounter{mybibstartvalue}{0}
%\setcounter{mybibstartvalue}{34}

\xpatchcmd{\thebibliography}{%
  \usecounter{enumiv}%
}{%
  \usecounter{enumiv}%
  \setcounter{enumiv}{\value{mybibstartvalue}}%
}{}{}

%%%%%%%%%%%%%%%%% END OF PREAMBLE %%%%%%%%%%%%%%%%


\renewcommand{\theequation}{\arabic{equation}}
\setcounter{equation}{0}

\renewcommand{\thesection}{\arabic{section}}
\setcounter{section}{0}

\begin{document} 

% Double-space the manuscript.

\baselineskip24pt

% Make the title.

\maketitle 

% Place your abstract within the special {sciabstract} environment.

\begin{sciabstract}
\end{sciabstract}

% In setting up this template for *Science* papers, we've used both
% the \section* command and the \paragraph* command for topical
% divisions.  Which you use will of course depend on the type of paper
% you're writing.  Review Articles tend to have displayed headings, for
% which \section* is more appropriate; Research Articles, when they have
% formal topical divisions at all, tend to signal them with bold text
% that runs into the paragraph, for which \paragraph* is the right
% choice.  Either way, use the asterisk (*) modifier, as shown, to
% suppress numbering.

\section{Method}

% In this section, we present an overview of the methods. In the
% supplementary text that follows, additional supporting information
% regarding the methods is provided, as well as figures and a table.

Our goal is to extract the decadal signal of water-mass change from
the historical temperature observations.  The historical-to-WOA
temperature difference is computed by interpolating the World Ocean
Atlas values onto the historical data locations. The historical-to-WOA
temperature difference profile is computed by a least-squares method
that accounts for contamination by measurement error and signals that
are not representative of the decadal-mean, basinwide-average
temperature. The contamination is assumed to have three parts: (1)
transient effects such as isopycnal heave due to internal waves or
mesoscale eddies, (2) irregular spatial sampling of each basin, and
(3) measurement or calibration error of the thermometer.  The expected
size of (1) is taken from the WGHC error estimates and depends upon
location.  Following \cite{Huang--2015:Heaving}, the variance due to
(2) is assumed to be $20\%$ that of (1), and the standard error due to
(3) is assumed equal to $0.14^{\circ}$C
\cite{Roemmich-Gould-2012:135}.


\subsection{Vertical profile of temperature difference}

The temperature difference between the historical observations and the
corresponding World Ocean Atlas (WOA) value is,
\begin{equation}
\Delta T(r_i) = T(r_i,t_w) - T(r_i,t_h),
\end{equation}
where $T(r_i,t_h)$ is the $i$th {\it historical} temperature
observation at location, $r_i$, and time, $t_h$, and $T(r_i,t_w)$ is
the WOA temperature at the same location. The observations are
combined into a vector,
\begin{equation}
{\boldsymbol \Delta}{\bf T} = \left(\begin{array}{c} \Delta T(r_1) \\
                                      \Delta T(r_2) \\ \vdots \\
                                      \Delta T(r_M) 
 \end{array} \right) .
\end{equation}
Temperature changes at a given pressure are assumed equivalent to potential temperature changes.

\subsection{Basinwide-average temperature profiles}

Our goal is to extract the decadal signal of water-mass change from
the historical temperature observations 
\begin{equation}
\overline{{\boldsymbol \Delta}{\bf T}} = \left(\begin{array}{c} \overline{\Delta \theta(z_1)}
                  \\\overline{\Delta \theta(z_2)} \\ \vdots \\ \overline{\Delta \theta(z)_K} 
 \end{array} \right),
\end{equation}
where we have defined a grid of $K$ depths.  Given knowledge of the
basinwide averages, one can make a prediction for each WOA$-$historical temperature difference,
\begin{equation}
{\boldsymbol \Delta}{\bf T} = {\bf V}\overline{{\boldsymbol \Delta}{\bf T}} + {\bf q},
\end{equation}
where ${\bf V}$ maps the basinwide mean onto the observational point
by noting the basin of the observations and vertical linear
interpolation, ${\bf q}$ is contamination by measurement error and
signals that are not representative of the decadal-mean,
basinwide-average temperature. The contamination is decomposed into
three parts,
\begin{equation}
{\bf q} = {\bf n}_T + {\bf n}_S + {\bf n}_M,
\end{equation}
where ${\bf n}_T$ is contamination by transient effects such as
isopycnal heave due to internal waves or mesoscale eddies, ${\bf n}_S$
is due to the irregular spatial sampling of each basin, and
${\bf n}_M$ is measurement or calibration error of the
thermometer. Note that no depth correction is made here, and
temperature differences may be biased toward warming.

The expected size of ${\bf n}_T$ is related to the energy in the
interannual and higher-frequency bands. We use estimates from the WOCE
Global Hydrographic Climatology \cite{Gouretski-Koltermann-2004:WOCE}
to quantify this error and its spatial pattern.  Errors that primarily
reflect an uncertainty due to a representativity error were previously
estimated in this climatology, where the magnitude of interannual
temperature variability is $1.6^{\circ}$C at the surface, decreasing
to $0.8^{\circ}$C below the mixed layer, and $0.02^{\circ}$C at 3000
meters depth. Inherent in their mapping is a horizontal lengthscale of
$L_{xy}^{T}=450$ km. This corresponds to a vertical lengthscale of
$L_z^{T}=450$ meters when applying an aspect ratio based upon mean
depth and lateral extent of the ocean. Their mapping is the degree of
error necessary to place the non-synoptic cruises of a 10-year time
interval into a coherent picture. Estimated errors are similar to
those of \cite{Wortham-Wunsch-2014:multidimensional}, who also note
that the spatial scales increase as the temporal scales
increase. Above 1300 meters depth, the aliased variability is
typically larger than the measurement error described below.

Next we describe the second moment matrix of temporal contamination,
${\bf R}_{TT} =<{\bf n}_T ({\bf n}_T)^T>$. Note that ${\bf n}_T$
depends on the difference of contamination during the two time
periods, $n_T(r_i) = \eta_T(r_i,t_w) - \eta_T(r_i,t_h)$, where
$\eta_T(r,t)$ is the difference between temperature at a given time
and the decadal average. The WGHC statistics give the error covariance
for $\eta_T(r,t_w)$ not $n_T(r)$. This covariance matrix is
reconstructed by first creating a correlation matrix,
\begin{equation}
  {\bf R}_{\rho} = \left(\begin{array}{ccccc}
\rho(0) & \rho(\delta) & \rho(2\delta) & \hdots & \\
\rho(\delta) & \rho(0) & \rho(\delta) & \hdots &  \\    
\rho(2\delta) & \rho(\delta) & \rho(0) & \hdots &  \\    
\vdots &   \vdots           &  \vdots       &    \ddots & \\
&              &         &     & \rho(0) \end{array} \right),
\end{equation}
where the autocorrelation function, $\rho(\delta)$, is given by a
Gaussian with a horizontal lengthscale of 450 km and a vertical
lengthscale of 450 meters. We derive the covariance matrix by pre- and
post-multiplying the correlation matrix,
${\bf R}_{\eta\eta} = {\boldsymbol \sigma}_\eta {\boldsymbol
  \sigma}_\eta^T \circ {\bf R}_{\rho} $, where
${\boldsymbol \sigma}_\eta$ is the vector of the standard deviation of
the WGHC interannual variability and $\circ$ is the Hadamard
product. Here the time interval of the historical cruises is about 20
years, or twice as long as the WOCE era. Due to the red spectrum of
ocean varabilility, the potential for aliased variability over this
longer time interval is increased. We roughly approximate this extra
energy with $T_{ratio} = 2$. Both the modern and historical intervals
have variability and are assumed to be statistically independent, and
thus, ${\bf R}_{TT} = 2 T_{ratio} {\bf R}_{\eta\eta}$.

We assume that the variance due to spatial water-mass variability,
i.e., ${\bf R}_{SS} =<{\bf n}_S ({\bf n}_S)^T>$, has a magnitude that
is $20\%$ that of the temporal variability as the local water-mass
variability on interannual scales is dwarfed by heaving motions
\cite{Huang--2015:Heaving}. The relevant parameter is
$S_{ratio} = 0.2$. These water-mass variations are assumed to have a
larger spatial scale ($L_{xy}^S = 2000$ km horizontally, $L_z^S = 1$
km vertically), as seen in an evaluation of water-mass fractions on an
isobaric surface \cite{Gebbie-Huybers-2010:Total}. Accounting for this
spatial variability has the potential to increase the final error of
our estimates by taking into account biases that may occur due to the
specific expedition tracks. Finally, we assume that the measurement
covariance, ${R_{MM}}$, is a matrix with the diagonal equal to the
observational uncertainty, $\sigma_{obs} = 0.14^{\circ}$C, squared
\cite{Roemmich-Gould-2012:135}.

We solve for the basinwide-average temperature profiles using a
weighted and tapered least-squares formulation that minimizes, 
\begin{equation}
\label{eq:qWq}
J = {\bf q}^T {\bf R}_{qq}^{-1} {\bf q} + {\bf m}^T {\bf S}^{-1} {\bf m},
\end{equation}  
where ${\bf R}_{qq}$ reflects the combined effect of the three types of errors (i.e., ${\bf R}_{qq} = {\bf R}_{TT} + {\bf R}_{SS} + {\bf R}_{MM}$). This least-squares weighting %(i.e., ${\bf W} = {\bf R}_{qelta}q}$) 
is chosen such that the solution coincides with the maximum likelihood estimate (assuming that the prior statistics are normally distributed and appropriately defined).
Only a weak prior assumption, reflected in the weighting matrix, ${\bf
  S}$, is placed on the solution, namely that the correlation
lengthscale is $L_z^{AVG} = 500$ m in the vertical, the variance is on the order of
($\sigma_S = 1^\circ$C$)^2$, and the expected value is $<\overline{{\boldsymbol \Delta}{\bf T}}>=0$.
The least-squares estimate is then,
\begin{equation}
\tilde{\overline{{\boldsymbol \Delta}{\bf T}}} = ({\bf V}^T {\bf R}_{qq}^{-1} {\bf V} + {\bf S}^{-1} )^{-1} {\bf V}^T {\bf R}_{qq}^{-1} {\boldsymbol \Delta}{\bf T}.
\end{equation}
 The error covariance of the estimate is,
\begin{equation}
  \label{eq:covariance17}
  \mbox{{\bf C}}_{{\tilde{\Delta T}}} = ({\bf V}^T {\bf R}_{qq}^{-1} {\bf V} + {\bf S}^{-1} )^{-1},
\end{equation}
where the standard error is
${\boldsymbol \sigma}_{\tilde{\Delta T}} =
\sqrt{\mbox{diag}(\mbox{{\bf C}}_{{\tilde{\Delta T}}})}$.  This method
also recovers the off-diagonal terms that correspond to the correlated
errors among different parts of the basinwide-average.

\subsection{Ocean heat content change}

Ocean heat content change, $\Delta {\cal H}$, is a linear function of
the temperature change and can be written as an inner vector product:
\begin{equation}
\Delta {\cal H} = {\bf h}^T \tilde{\overline{{\boldsymbol \Delta}{\bf T}}},
\end{equation}
where ${\bf h}$ is a vector containing coefficients related to ocean
heat capacity, seawater density, the representative area of the Indian
Ocean, and the integration of temperature change over the vertical
dimension. Here we integrate to a depth of $z\star = 700$ m so that we
obtain heat content change from the sea surface to this depth. The
Indian Ocean area is assumed to be equal to 15\% of the global ocean
area at all depths (ignoring the hypsometric effect).
  
The error covariance of $\Delta {\cal H}$ is an outer product, 
\begin{equation}
\mbox{{\bf C}}_{H} = <({\bf h}^T \tilde{\overline{{\boldsymbol \Delta}{\bf T}}})({\bf h}^T \tilde{\overline{{\boldsymbol \Delta}{\bf T}}})^T>,
\end{equation}
where $<>$ refers to the expected value. ${\bf C}_H$ is a scalar like $\Delta {\cal H}$.
Rearranging this equation, we obtain, 
\begin{equation}
\mbox{{\bf C}}_{H} =  {\bf h}^T \mbox{{\bf C}}_{{\tilde{\Delta T}}} {\bf h},
\end{equation}
where $\mbox{{\bf C}}_{{\tilde{\Delta T}}}$ is known from the
calculation of the previous section. The standard error of the heat
content change is the square root of $\mbox{{\bf C}}_{H}$.

\subsection{Historical thermometer corrections}

The historical observations are corrected for the
compressibility of the thermometers \cite{Tait--1882:Pressure}.
Subsequent to the cruise, a bias of no more than 0.04$^\circ$C per the
equivalent of a kilometer of depth was estimated in pressure-tank
experiments \cite{Tait--1882:Pressure}. In order to guard against
biases that would predispose our analysis toward deep-Pacific cooling,
temperature is adjusted to be 0.04$^\circ$C cooler per kilometer of
depth, in keeping with previous analyses
\cite{Roemmich-Gould-2012:135}. With this adjustment, the amount of
Pacific cooling is diminished between the historical and recent eras.

We correct for pressure-compression effects altering the mercury level
in the max-min thermometers used in the {\it historical} expedition.
Subsequent to the cruise, a bias of no more than 0.04$^\circ$C per the
equivalent of a kilometer of depth was estimated in pressure-tank
experiments\cite{Tait--1882:Pressure}. In order to guard against
biases that would predispose our analysis toward deep-Pacific cooling,
we adjust temperature to be 0.04$^\circ$C cooler per kilometer of
depth, in keeping with previous
analyses\cite{Roemmich-Gould-2012:135}. With this adjustment, the
amount of Pacific cooling is diminished between the historical and WOA
eras.

There are other issues related to the depth of the observation.  In
abyssal locations with temperature inversions, the max-min thermometer
will lead to a cold bias that may obscure trends in temperature since
the historical expeditions were completed. For this reason, points
could be (but are not) eliminated from the analysis.

Results are labeled with ` tait` when pressure corrections are
applied. 

\newpage

\bibliography{HistoricalIndianOcean}

\bibliographystyle{Science}



\clearpage

% \noindent {\bf Fig. Instructions.} Please do not use figure environments to set
% up your figures in the final (post-peer-review) draft, do not include graphics in your
% source code, and do not cite figures in the text using \LaTeX\
% \verb+\ref+ commands.  Instead, simply refer to the figure numbers in
% the text per {\it Science\/} style, and include the list of captions at
% the end of the document, coded as ordinary paragraphs as shown in the
% \texttt{scifile.tex} template file.  Your actual figure files should
% be submitted separately.

% \clearpage



% \begin{figure}%[htbp]
% \begin{center}
% \includegraphics[scale=0.9]{suppfigs/Joffset_31-Oct-2017.eps}
% \noindent{{\bf Fig.~S1. Optimal offset of proxy and instrumental surface temperature datasets.} The cost function, $J$, calculated as the misfit between a Common Era simulation run from equilibrium at 15 C.E. (EQ-0015) and the basinwide-average WOA minus {\it historical} temperature differences. The boundary conditions are re-derived as a function of the offset, where positive values of the offset indicate that the instrumental surface temperature product is shifted to more positive values relative to the proxy record. The optimal offset is $0.067^{\circ}$C. The expected value of $J$ is normalized to 1 based on the data error statistics.}
% \end{center} 
% \end{figure}

\end{document}

%%% Local Variables:
%%% mode: latex
%%% TeX-master: t
%%% End:
